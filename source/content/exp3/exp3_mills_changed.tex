\documentclass{article}

\usepackage{amsmath}
\usepackage{comment}

\title{Physics 6A -- Experiment 3\\ {\Large Additional Credit correction}
	\\ {\normalsize Version 1}}
\date{}
\begin{document}

\maketitle


There are two additional credit assignments.  You may wish to do the second shorter one first to take a break from the air track measurements.

\begin{comment}
\subsection*{ADDITIONAL CREDIT PART 1: DIFFERENT MASS GLIDER (5 mills)}

Repeat the entire experiment with the second glider, which has a different mass.  Complete another Excel sheet with the data and chart with equation.  Below the chart, type in the mass of the glider obtained experimentally from the slope, the mass obtained by weighing, and the experimental error.  Be sure your chart has a title and the axes are labeled.
\end{comment}

\subsection*{ADDITIONAL CREDIT PART 1: DIFFERENT GLIDER SETUP (5 mills)}

Repeat the experiment with the same glider.
However, this time you will NOT keep the total mass of the system constant.
Place masses on the hanger and obtain the acceleration of the glider as before, 
but this time \emph{do not place any additonal mass on the glider}.

If $M$ is the mass of the glider and $m$ is the mass of both the hanger and the
small weights, the various masses involved should be as
follows:
\begin{align*}
	\begin{array}{c|c|c}
%		\text{Run} & 
%		\text{Mass on hanger} & \text{Mass on glider}
		\text{Hanging mass} & \text{Gliding mass}
		& \text{Total mass}
		\\
		\hline
%		1 & 
		m & M+3m & M+4m\\
%		2 & 
		2m & M+2m & M+4m\\
%		3 & 
		3m & M+m & M+4m\\
%		4 & 
		4m & M & M+4m
	\end{array}
\end{align*}

This differs from the masses involved in the original experiment:
\begin{align*}
	\begin{array}{c|c|c}
%		\text{Run} & 
%		\text{Mass on hanger} & \text{Mass on glider}
		\text{Hanging mass} & \text{Gliding mass}
		& \text{Total mass}
		\\
		\hline
%		1 & 
		m & M & M+m\\
%		2 & 
		2m & M & M+2m\\
%		3 & 
		3m & M & M+3m\\
%		4 & 
		4m & M & M+4m
	\end{array}
\end{align*}

Complete another Excel sheet with the data.
Now combine your previous data with your new data, to plot both sets of data on
the same chart.
Accomplish this by making 3 columns -- Hanging weight, Acceleration in original
experiment, and acceleration in this additional credit part.
Then select all 3 columns, and insert a scatter chart.

Below the chart, type in the mass of the glider obtained experimentally from the slope, the mass obtained by weighing, and the experimental error.  
Be sure your chart has a title and the axes are labeled.


\begin{comment}
\subsection*{ADDITIONAL CREDIT PART 2: DIFFERENT METHODS OF STARTING THE GLIDER (2 mills)}

When the air pump is first turned on, there may be some friction between the track and the glider, since it takes a few moments for the pressure to reach its maximum value.  Try several runs in which the glider is released a few seconds after the maximum air pressure has been attained, and several runs in which air causes the glider to ``float'' up the track before it begins to accelerate downward.  Is there any systematic difference in the data obtained from the two methods?  Which method produces results that agree more closely with Newton's Second Law?  Explain what you are doing, and record your measurements and conclusions.
\end{comment}

\subsection*{ADDITIONAL CREDIT PART 2: FREE-FALLING PICKET FENCE \bf{(2 mills)}}

Here you will need a photogate, a picket fence, and a rag box to catch the falling fence.  The picket fence is a strip of clear plastic with evenly spaced black bars.  When the fence is dropped through a photogate, the light beam is interrupted by the bars; since the fence accelerates while falling, the bars interrupt the beam with increasing frequency.  The software calculates the distance fallen, as well as the corresponding velocity and acceleration.  This acceleration should be constant and equal in magnitude to \(g\).

% \begin{center} \includegraphics*[width=0.4\textwidth]{imgs/6labs/6Alab/6Aexp3/exp_fig7_fx.jpg} \end{center}

Double-click on ``Photogate Plus Picket Fence'' in the list of sensors, and insert the physical plug of the photogate into the appropriate digital channel of the interface.  Drag the graph icon to the picket fence icon in the steup picture, and set it to plot position, velocity, and acceleration as separate graphs aligned vertically.  Arrange the photogate in such a way that the falling picket fence will be caught by the rag box on the floor.

Click ``Start'', drop the picket fence through the photogate, and click ``Stop''.  Since the entire motion occurs within a fraction of a second, you may not see much on the graph.  Expand the time scale using the ``Scale-to-fit'' button, and click the zoom box at the upper-right corner so the graph window fills the screen.  Again, the acceleration graph may look ragged.  Use the statistics button to find the mean value of the acceleration.  Record the magnitude of the acceleration due to gravity with experimental error below.  You can print your sheet of three graphs to keep for your records.

Experimental \(g\) = ?

Experimental error = ?

This short experiment illustrates the power of computer measurement.  It appears easy, but be sure you understand what is happening in the measurement and can explain the results to your TA.

\end{document}
