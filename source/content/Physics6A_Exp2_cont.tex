\fancyhead[C]{Physics 6A Lab \rule[-1ex]{0pt}{0pt}\(\mid\) Experiment 2}

\vspace{-5ex}\part*{Kinematics}

\subsection*{APPARATUS}

\vspace{-0.5ex}
\vphantom{.}
\squishlist
\item Computer and Pasco interface
\item Motion sensor (sonic ranger)
\item Air track, glider with reflector, block to tilt track
\item Calipers to measure thickness of block
\item Pendulum arrangement
\squishend

\subsection*{INTRODUCTION}

In this experiment, you will produce position, velocity, and acceleration graphs of your own movements, as well as that of a glider on an air track, using a motion sensor.  A motion sensor (sometimes called a sonic ranger) measures the distances to objects by repeated reflection of ultrasonic sound pulses.  The software included with this device takes the first and second derivatives of the position measurements to calculate the velocity and acceleration, respectively.
\begin{center} \includegraphics*[width=0.6\textwidth]{imgs/6labs/6Alab/6Aexp2/6A_exp2_set-up_new.jpg} \end{center}

To determine distances, the motion sensor emits and receives ultrasound pulses at a frequency of approximately 50 kHz.  Since the speed of ultrasound in air at room temperature is known, the software calculates these distances by measuring the time required for the pulse to reflect from an object and return to the sensor.  This process is similar to how a bat ``sees'' using ultrasound, as well as how a Polaroid autofocus camera determines the distance to an object in order to focus properly.

The ultrasonic sound emitted by the motion sensor spreads about 15{\degree} off axis.  Keep this in mind as you design your experiments.  The sensor does not work for objects closer than 0.4 meters.  (Some of the newer motion sensors have adjustable width beams and will measure objects as close as 15 centimeters.)

The clicking noise made by the motion sensor is not ultrasound, but a by-product of the mechanism that produces the ultrasound.  Most people cannot hear the frequencies emitted by the sensor.  If you place your ear near the device, however, you may be able to ``feel'' the pressure pulse of sound against your eardrum.

\subsection*{INITIAL SETUP}

\begin{enumerate}[label=\arabic*.]

\item Turn on the signal interface and computer, plug the motion sensor into the interface (inserting the yellow-banded plug into digital channel 1 and the other plug into digital channel 2), and call up Capstone.

\item Choose ``Table \& Graph'', then under Hardware Setup, click on channel 1 of the interface.  Choose ``Motion Sensor II''.
\begin{center} \includegraphics*[width=0.6\textwidth]{imgs/6labs/6Alab/6Aexp2/6A-exp2_fig2_new.jpg} \end{center}

\item Click ``Select Measurement'' on the \(y\)-axis of the graph and choose ``Position (m)''.  There is a tool at the top of the graph that says ``Add new plot area to the Graph display''.  Click on this tool twice to display a total of three graphs.

\item For the second graph, set ``Velocity (m/s)'' on the \(y\)-axis.  For the third graph, set ``Acceleration $(m/s^2)$'' on the \(y\)-axis.

\item Click ``Record'' to activate the motion sensor.  You should hear a series of rapid clicks.  Test out the motion sensor by holding a book facing, and approximately 0.5 meter away from, the sensor.  (The sensor does not measure distances closer than 0.4 meter.)  Move the book to a distance of 1.5 meters from the sensor and then back to 0.5 meter, click ``Stop'', and observe the graph.  You may want to click the ``Scale axes to show all data'' button at the left end of the tool bar above the graphs.  If you do not obtain a clean trace, try again, being careful to keep the book directly in front of the sensor.

\end{enumerate}

\subsection*{PROCEDURE}


\begin{comment}
% Warm-ups with position and velocity graphs
%	Including reporoducing trapezoidal x vs t and v vs t graphs
% Seem to distract from main part of procedure
{{{

In steps 1 -- 5, you will need to move in a straight line away from the motion sensor for a distance of 2 meters.  Begin by removing any objects that may hamper your motion or otherwise reflect the signal from the sensor.  Arrange the computer monitor so you can see the screen while traveling backward, and be sure your path is clear.  Obtain a graph on the screen that shows only the position data.  (If necessary, delete your graph from above, start with a new graph, and drag over the position data.)

\begin{enumerate}[label=\arabic*.]

\item Start approximately 0.5 meter from the motion sensor, click on ``Start'', and walk away steadily while holding a book facing the sensor.  Sketch the position-versus-time graph made by the computer in the ``Data'' section that follows in the lab manual.  To discard a data set, click on its run number on the graph to highlight it, hit delete, and click ``OK''.  Note that you can obtain several position curves in different colors on the same graph.

\item How would you expect the graph in part 1 to change if you walked away from the sensor more rapidly?  First postulate an answer; then record and execute this motion.  Sketch the position-versus-time graph made by the computer in the ``Data'' section.

\item In the ``Data'' section, sketch velocity-versus-time graphs based on the position graphs you depicted in parts 1 and 2.  Compare these with the actual velocity graphs made by the computer (which you can obtain by dragging the velocity data to the graph).

\item Execute the motion described qualitatively by the position-versus-time graph shown below.  (That is, move in such a way that you produce a computer graph like the one below.)  Ask your TA to check it.
\begin{center} \includegraphics*{imgs/6labs/6Alab/6Aexp2/exp_2_fig2_fx.jpg} \end{center}

\item Obtain a velocity graph on the screen, and move in such a way that you produce a computer graph described qualitatively by the velocity-versus-time graph shown below.  Ask your TA to check it.  You may find the velocity graph more difficult to reproduce than the position graph.  Have each lab partner attempt the motion about five times, and display the best result.  If you are not able to obtain the approximate shape, ask your TA to demonstrate, and try to repeat the motion.
\begin{center} \includegraphics*{imgs/6labs/6Alab/6Aexp2/exp2_fig3_fx.jpg} \end{center}

\end{enumerate}
}}}
\end{comment}

%In steps 6 -- 9, you will use an air track.

%\begin{enumerate}[start=6]
\begin{enumerate}

\item Turn on the air track and level it by adjusting the leveling screw such that a glider on the track has no apparent tendency to move in either direction.  Since air tracks are often bowed in the middle, place the glider at several different positions on the track to verify that it is as level as possible.  Attach a reflector to the glider.
\begin{center} \includegraphics*[width=0.4\textwidth]{imgs/6labs/6Alab/6Aexp2/6A-EXP2_gilder.jpg} \end{center}

Arrange the sensor so it points slightly downward and can follow the glider along the entire length of track, as shown below.  Place your eye at the position you want the sensor to point.  Look into its reflective face, and adjust the sensor until you see your reflected image.  The sensor is now directed at your eye position.
\begin{center} \includegraphics*[width=0.8\textwidth]{imgs/6labs/6Alab/6Aexp2/exp2_fig5fx_2.jpg} \end{center}

Tilt the end of the track up by placing the small block under the leveling screw.  Give the glider a small hand push up the track.  The glider should slow as it moves up, stop momentarily before reaching the top end, and then coast back down the track.  After experimenting with several trials, predict the position, velocity, and acceleration of the glider as it moves along the track.

Now set your computer to produce all three graphs aligned vertically.  Record the motion of the glider moving up, slowing, stopping, and moving back down.

\begin{center}
\begin{tabular}{|p{14cm}|}
\hline\tstrut
Practice the attitude that you are going to make the experiment work well and produce a good position graph.  If your initial graphs are not clean, adjust the position and direction of the motion sensor, adjust the tilt of the reflector on the glider, etc., until you get a good data set.  (Your acceleration graph may still look ragged.)  When you have a graph with clean data on the screen, compare this with your predictions.  You can print it out to keep for your records.  \bstrut\\
\hline
\end{tabular}
\end{center}

\item Use the Vernier caliper (see instructions below) to measure the height \(h\) of the block which tilts the track, and use the ruler built into the track to measure the distance \(D\) along the track between the track supports.  The sine of the tilt angle \(\alpha\) is equal to \(h/D\); knowledge of \(h\) and \(D\) allows you to determine \(\alpha\).  As the glider moves along the track, Newton's Second Law predicts that its acceleration should be constant and equal in magnitude to \(g\sin\alpha\).  Calculate the value of this acceleration.  This is your theoretical prediction for the acceleration of the cart.

\item Acceleration graphs often look ragged, as small errors accumulate when the software takes the second derivative of the position data.  We will use the method of linear regression to determine the the acceleration of the cart.  Notice the linearly increasing portion of your velocity graph.  Select this region of interest by first clicking on the velocity graph then clicking on the ``Highlight range of points in active data'' tool.  A rectangle will appear.  Drag and extend this rectangle over the linearly increasing portion of your data.  Click inside the rectangle to highlight the data.

\item Click the drop-down arrow of the ``Apply selected curve fits to active...'' tool and choose ``Linear''.  Click this tool again to bring up a display box.  You should see a best-fit line appear on top your selected data and a box should pop up that tells you the slope and  \(y\)-intercept of this best-fit line.  The slope of this line is your experimentally measured acceleration of the cart.  Write this value down for your analysis. 

\item Another way to measure the acceleration is by looking at the acceleration graph.  In general your graph should be flat (and possibly a a bit ragged).  We want to find the average of these data points.  To do this, select the data points of interest using the ``Highlight range of points in active data'' tool.  Click the ``Display selected statistics for active...'' tool.  This will display the mean value of the points you selected.  Compare this to the value you obtained by making a best-fit line.

\item Calculate the percentage error between the experimental results \(r_{\textrm{exp}}\) obtained in steps 4 and 5, and the theoretical value \(r_{\textrm{th}}\) determined in step 2:

percentage error = \( (100\%) \times \left(\mid r_{\textrm{exp}} - r_{\textrm{th}} \mid\right) / r_{\textrm{th}} . \)

The absolute value is taken because we are not interested in the sign of the difference.  Typical experimental accuracies in an undergraduate lab range from 3 -- 5\%, although some quantities can be measured much more accurately (and some much less!).

\end{enumerate}

\subsection*{VERNIER CALIPER}

The \textbf{Vernier caliper} is designed to provide a highly precise measurement of length.  The numbers in parentheses refer to those found in the figure below.
\begin{center} \includegraphics*[width=0.65\textwidth]{imgs/6labs/6Alab/6Aexp2/6A-EXP2-fig7_webtext.jpg} \end{center}

The \textbf{outside jaws} (1) are used to measure around the exterior of an object.  The \textbf{inside jaws} (2) are used to measure inside the holes of an object.  The \textbf{depth gauge} (3) is used to measure the depth of holes.

The \textbf{Vernier} (4) is used to divide further the \textbf{metric scale} (5) or the \textbf{English scale} (6) down to 0.01 centimeter or 1/128 inch, respectively.  Its operation is described in detail below.

The \textbf{thumbwheel} (7) is used to open and close the jaws.

The \textbf{locking screw} (8) is used to lock the jaws in position.

The \textbf{Vernier} is a device used for estimating fractional parts of distances between two adjacent divisions on a \textbf{scale}.  The \textbf{Vernier scale} subdivides each \textbf{main scale} division into as many parts as there are divisions on the scale.

The \textbf{Vernier} and \textbf{main scales} are placed in contact with each other.  The \textbf{main scale} is read to the nearest number of whole divisions, while the zero on the \textbf{Vernier scale} serves as the index (i.e., the line on the extreme left).  One should then estimate the fractional part of the \textbf{main scale} reading as a check of the more accurate reading to be made with the aid of the \textbf{Vernier scale}.

The \textbf{metric scale} is divided into tenths of a centimeter (i.e., millimeters).  The \textbf{Vernier scale} is divided into 10 divisions, thus representing hundredths of a centimeter.  \ul{At any given setting, the marks on the \textbf{Vernier} and \textbf{main scales} coincide at only one point.}  The mark on the \textbf{Vernier scale} which coincides most closely with a corresponding mark on the \textbf{main scale} represents the Vernier reading.  The figure below should help clarify this.
\begin{center} \includegraphics*[width=0.5\textwidth]{imgs/6labs/6Alab/6Aexp2/caliper_alt_2.jpg} \end{center}

\subsection*{DATA}

This section is to help you organize your data and guide you through the calculation.

\begin{enumerate}[label=\arabic*.]

\begin{comment}
% Data for warm-up portion of procedure
{{{
\item Sketch the position graph made by the computer using the graph paper at the end of this workbook.

\item Sketch the position graph made by the computer using the same sheet of graph paper.

\item Sketch the predicted velocity graphs using the same sheet of graph paper.
}}}
\end{comment}

%\setcounter{enumi}{8}
\item Mean value of acceleration = \ul{~~~~~~~~~~~~~~~~~~~~~~~~~~~~~~~~~~~~~~~~~~~~~}

Slope of velocity graph (best-fit line) = \ul{~~~~~~~~~~~~~~~~~~~~~~~~~~~~~~~~~~~~~~~~~~~~~}


\begin{center}
\begin{tabular}{|p{14cm}|}
\hline\tstrut
Always record units with your results.  \bstrut\\
\hline
\end{tabular}
\end{center}

\begin{comment} Always record units with your results. \end{comment}

\end{enumerate}

\subsection*{CALCULATIONS}

%\begin{enumerate}[start=7]
\begin{enumerate}[start=2]

\item \(g\sin\alpha\) = \ul{~~~~~~~~~~~~~~~~~~~~~~~~~~~~~~~~~~~~~~~~~~~~~}

% %\setcounter{enumi}{9}
% \setcounter{enumi}{4}
\item  Percentage error for mean value of acceleration = \ul{~~~~~~~~~~~~~~~}

Percentage error for slope of velocity graph (best-fit line) = \ul{~~~~~~~~~~~~~~~}


\end{enumerate}

\subsection*{QUESTIONS}

\begin{enumerate}[label=\alph*.]

\item What does calculus tell us about the relationship between the position and velocity graphs?

\item Suppose you were able to arrange the motion sensor to measure and plot the position, velocity, and acceleration of a ball thrown vertically upward into the air.  Neglecting air resistance, how would these graphs compare with those of the glider experiment?  Elucidate any differences.

\end{enumerate}


\subsection*{ADDITIONAL CREDIT: Kinematics Graphs (3 mills)}
In the steps below, you will need to move in a straight line away from the
motion sensor for a distance of roughly 2 meters.  
Begin by removing any objects that may hamper your motion or otherwise reflect the signal from the sensor.  Arrange the computer monitor so you can see the screen while traveling backward, and be sure your path is clear.  
%Obtain a graph on the screen that shows only the position data.  (If necessary, delete your graph from above, start with a new graph, and drag over the position data.)

\begin{enumerate}[label=\arabic*.]

\item 
	While holding a book facing the sensor, 
	move in such a way that you qualitatively reproduce the 
	position-versus-time graph pictured below.
	(That is, move in such a way that you produce a
	computer graph shaped roughly like the one below.  
	However, don't worry about getting any of the numbers to match up -- we're
	only interested in reproducing the shape here.)
\begin{center} \includegraphics*{imgs/6labs/6Alab/6Aexp2/exp_2_fig2_fx.jpg} \end{center}

\item 
	Now move in such a way that you produce a computer graph described 
	qualitatively by the velocity-versus-time graph shown below.  
	You may find the velocity graph more difficult to reproduce than the position graph.  Have each lab partner attempt the motion a few times, and display the best result.  If you are not able to obtain the approximate shape, ask your TA to demonstrate, and try to repeat the motion.
\begin{center} \includegraphics*{imgs/6labs/6Alab/6Aexp2/exp2_fig3_fx.jpg} \end{center}

\end{enumerate}


\subsection*{ADDITIONAL CREDIT: PENDULUM MOTION (3 mills)}

Arrange the motion sensor to track a swinging pendulum.  For a pendulum bob,
use a block with a flat side facing the sensor.  Set the pendulum into motion
with small-amplitude oscillations (so the pendulum swings only a few inches), 
carefully positioning and aligning the sensor so it tracks this motion and produces smooth curves.  
Display a graph showing position, velocity, and acceleration.  Be prepared to discuss which mathematical function describes each graph, as well as how the graphs are related, and ask the TA to check your work.

Note:  It is very common for students doing this experiment to position the
motion sensor in such a way that the pendulum is out of view of the motion
sensor for at least part of the motion.  
As stated above, the pendulum should move in a \emph{smooth} curve -- 
if your curve is mostly smooth but has spikes or bumps at the uppermost or
lowermost points, then you likely need to reposition the sensor.

