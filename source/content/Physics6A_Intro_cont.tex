\fancyhead[C]{Physics 6A Lab \rule[-1ex]{0pt}{0pt}}

\vspace{-1ex}\part*{Introduction}

\subsection*{PURPOSE}

The laws of physics are based on experimental and observational facts.  Laboratory work is therefore an important part of a course in general physics, helping you develop skill in fundamental scientific measurements and increasing your understanding of the physical concepts.  It is profitable for you to experience the difficulties of making quantitative measurements in the real world and to learn how to record and process experimental data.  For these reasons, successful completion of laboratory work is required of every student.

\subsection*{PREPARATION}

\begin{comment}
Read the lab manual for the assigned experiment before coming to the lab.  Since the experiments must be finished during the lab sessions, this will greatly speed up your lab work, and in this lab series, you may leave early if you finish the work.  Although we dislike testing you, if your TA suspects that you haven't read the lab manual before coming to lab, he or she may ask you directly, or ask you a few simple questions about the experiment.  If you cannot answer satisfactorily, you may lose mills (see below).
\end{comment}
Read the assigned experiment in the manual before coming to the laboratory.  Since each experiment must be finished during the lab session, familiarity with the underlying theory and procedure will prove helpful in speeding up your work.  Although you may leave when the required work is complete, there are often ``additional credit'' assignments at the end of each write-up.  The most common reason for not finishing the additional credit portion is failure to read the manual before coming to lab.  We dislike testing you, but if your TA suspects that you have not read the manual ahead of time, he or she may ask you a few simple questions about the experiment.  If you cannot answer satisfactorily, you may lose mills (see below).

\subsection*{RESPONSIBILITY AND SAFETY}
\begin{comment} \subsection*{ORDERLINESS AND SAFETY} \end{comment}

Laboratories are equipped at great expense.  You must therefore exercise care in the use of equipment.  Each experiment in the lab manual lists the apparatus required.  At the beginning of each laboratory period check that you have everything and that it is in good condition.  Thereafter, you are responsible for all damaged and missing articles.  At the end of each period put your place in order and check the apparatus.  By following this procedure you will relieve yourself of any blame for the misdeeds of other students, and you will aid the instructor materially in keeping the laboratory in order.

The laboratory benches are only for material necessary for work.  Food, clothing, and other personal belongings not immediately needed should be placed elsewhere.  A cluttered, messy laboratory bench invites accidents.  Most accidents can be prevented by care and foresight.  If an accident does occur, or if someone is injured, the accident should be reported immediately.  Clean up any broken glass or spilled fluids.

\begin{comment}
\subsection*{ORDERLINESS AND DISCIPLINE}
The laboratory benches are for work materials only. Clothing and other personal belongings should be placed elsewhere. A cluttered, messy bench invites accidents, most of which can be prevented by care and foresight. Any accident or injury should be reported immediately, and any broken glass or spilled fluids should be cleaned up promptly.
\end{comment}

\subsection*{FREEDOM}
\begin{comment} \subsection*{FREEDOM AND RESPONSIBILITY} \end{comment}

You are allowed some freedom in this laboratory to arrange your work according to your own taste. The only requirement is that you complete each experiment and report the results clearly in your lab manual.  We have supplied detailed instructions to help you finish the experiments, especially the first few.  However, if you know a better way of performing the lab (and in particular, a different way of arranging your calculations or graphing), feel free to improvise.  Ask your TA if you are in doubt.

\begin{comment}
Laboratories are equipped at great expense.  You must therefore exercise care in the use of equipment. Each experiment in the lab manual contains a list of required apparatus. At the beginning of each session, check that all equipment is present and in good condition. Thereafter, you are responsible for any missing or damaged articles. At the end of the session, put your station in order and check the apparatus. By following this procedure, you will relieve yourself of any responsibility for the misdeeds of other students, and will also aid the instructor materially in keeping the laboratory in order.
\end{comment}

\subsection*{LAB GRADE}

Each experiment is designed to be completed within the laboratory session.  Your TA will check off your lab manual and computer screen at the end of the session.  There are no reports to submit.  The lab grade accounts for approximately 15\% of your course total.  Basically, 12 points (12\%) are awarded for satisfactorily completing the assignments, filling in your lab manual, and/or displaying the computer screen with the completed work.  Thus, we expect every student who attends all labs and follows instructions to receive these 12 points.  If the TA finds your work on a particular experiment unsatisfactory or incomplete, he or she will inform you.  You will then have the option of redoing the experiment or completing it to your TA's satisfaction.  In general, if you work on the lab diligently during the allocated two hours, you will receive full credit even if you do not finish the experiment.

Another two points (2\%) will be divided into tenths of a point, called ``mills'' (1 point = 10 mills).  For most labs, you will have an opportunity to earn several mills by answering questions related to the experiment, displaying computer skills, reporting or printing results clearly in your lab manual, or performing some ``additional credit'' work.  When you have earned 20 mills, two more points will be added to your lab grade.  Please note that these 20 mills are additional credit, not ``extra credit''.  Not all students may be able to finish the additional credit portion of the experiment.

The one final point (1\%), divided into ten mills, will be awarded at the discretion of your TA.  He or she may award you 0 to 10 mills at the end of the course for special ingenuity or truly superior work.  We expect these ``TA mills'' to be given to only a few students in any section.  (Occasionally, the ``TA mills'' are used by the course instructor to balance grading differences among TAs.)

If you miss an experiment without excuse, you will lose two of the 15 points.  (See below for the policy on missing labs.)  Be sure to check with your TA about making up the computer skills; you may be responsible for them in a later lab.  Most of the first 12 points of your lab grade is based on work reported in your manual, which you must therefore bring to each session.  Your TA may make surprise checks of your manual periodically during the quarter and award mills for complete, easy-to-read results.  If you forget to bring your manual, then record the experimental data on separate sheets of paper, and copy them into the manual later.  However, if the TA finds that your manual is incomplete, you will lose mills.

In summary:
\begin{center}\begin{tabular}{ p{3.5cm} l }
 Lab grade =     & (12.0 points)   \\\tstrut\bstrut
                 & \(-\) (2.0 points each for any missing labs)  \\\tstrut\bstrut
                 & + (up to 2.0 points earned in mills of ``additional credit'')  \\\tstrut\bstrut
                 & + (up to 1.0 point earned in ``TA mills'')  \\\tstrut\bstrut
 Maximum score = & 15.0 points   \\\tstrut\bstrut
\end{tabular}
\end{center}

Typically, most students receive a lab grade between 13.5 and 14.5 points, with the few poorest students (who attend every lab) getting grades in the 12s and the few best students getting grades in the high 14s or 15.0.  There may be a couple of students who miss one or two labs without excuse and receive grades lower than 12.0.

How the lab score is used in determining a student's final course grade is at the discretion of the individual instructor. However, very roughly, for many instructors a lab score of 12.0 represents approximately B\(-\) work, and a score of 15.0 is A+ work, with 14.0 around the B+/A\(-\) borderline.

\subsection*{POLICY ON MISSING EXPERIMENTS}

\begin{enumerate}[label=\arabic*.]

\item In the Physics 6 series, each experiment is worth two points (out of 15 maximum points).  If you miss an experiment without excuse, you will lose these two points.

\item The equipment for each experiment is set up only during the assigned week; you cannot complete an experiment later in the quarter.  You may make up no more than one experiment per quarter by attending another section during the same week and receiving permission from the TA of the substitute section.  If the TA agrees to let you complete the experiment in that section, have him or her sign off your lab work at the end of the section and record your score.  Show this signature/note to your own TA.

\item (At your option) If you miss a lab but subsequently obtain the data from a partner who performed the experiment, and if you complete your own analysis with that data, then you will receive one of the two points.  This option may be used only once per quarter.

\item A written, verifiable medical, athletic, or religious excuse may be used for only one experiment per quarter.  Your other lab scores will be averaged without penalty, but you will lose any mills that might have been earned for the missed lab.

\item If you miss three or more lab sessions during the quarter for any reason, your course grade will be Incomplete, and you will need to make up these experiments in another quarter.  (Note that certain experiments occupy two sessions.  If you miss any three sessions, you get an Incomplete.)

\end{enumerate}
