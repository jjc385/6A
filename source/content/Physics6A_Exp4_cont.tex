\fancyhead[C]{Physics 6A Lab \rule[-1ex]{0pt}{0pt}\(\mid\) Experiment 4}

\vspace{-5ex}\part*{Conservation of Energy}

\subsection*{APPARATUS}

\noindent\textit{Shown in the diagram and picture below (both with a top-down view):}
\squishlist
\item Air track, springs and bracket, thread, glider
\item Smart pulley and mount
\item Dumb pulley (on the same mount)
\squishend
\begin{center} \includegraphics*[width=0.8\textwidth]{imgs/6labs/6Alab/6Aexp4/6A_exp4_fig1_fx.jpg} \end{center}
\begin{center} \includegraphics*[width=0.8\textwidth]{imgs/6labs/6Alab/6Aexp4/6A_EXP4-fig2a_pdf.jpg} \end{center}
\noindent\textit{Not shown in the images above:}
\squishlist
\item Computer and Pasco interface
\item Scale and weight set
\item Mass hanger
\squishend

\begin{comment}
\subsection*{INTRODUCTION}
\end{comment}

\subsection*{THEORY}

In this experiment, you will test the Law of Energy Conservation by monitoring an oscillating air track glider connected to springs by a thread which passes over a smart pulley.

Consider a glider of mass \(M\) on a nearly frictionless air track.  As the glider oscillates back and forth, there is a continuous exchange of mechanical energy between two forms: \ul{kinetic} energy contained in the moving glider, and \ul{potential} energy stored in the stretched or compressed springs.  The Law of Energy Conservation tells us that the \ul{total} mechanical energy of the system (i.e., the sum of the kinetic and potential terms) remains constant in time.  In reality, however, a small amount of mechanical energy may be lost to friction.

\subsection*{INITIAL SETUP}

\begin{enumerate}[label=\arabic*.]

\item Turn on the air track and level it by adjusting the leveling screw such that a glider on the track has no apparent tendency to move in either direction.  Since air tracks are often bowed in the middle, place the glider at several different positions on the track to verify that it is level.

\item Weigh the glider to obtain its mass, and record this value in the ``Data'' section.

\item Assemble the springs, glider, smart pulley, and thread as shown in the figure below.  One spring is attached to a bracket at the end of air track and to one end of the glider.  The other end of the glider is attached to a second spring by a thread which passes over the smart pulley.  The second spring is then attached to the bracket.  The springs should be tensioned so that you can get an end-to-end glider motion over a distance of at least 40 cm without either spring being completely compressed.

\item Mount a dumb pulley (which has low friction, but is not connected to the computer) on the same fixture that holds the smart pulley, albeit in a vertical plane.  Using a long piece of thread, connect the mass hanger to the glider.  Make sure the thread passes over the dumb pulley so that weights added to the hanger will displace the glider.
\begin{center} \includegraphics*[width=0.6\textwidth]{imgs/6labs/6Alab/6Aexp4/6A-exp4_fig1_fix.jpg} \end{center}

\end{enumerate}

\subsection*{PROCEDURE PART 1: MEASURING THE SPRING CONSTANT \(k\)}

\begin{enumerate}[label=\arabic*.]

\item Our first task is to measure \(k\), the force constant in \(F = -kx\) of a Hooke's Law spring.  Call up a blank Excel worksheet and prepare three columns to record the total mass in the hanger (including the mass of the hanger itself) in grams, its weight in newtons, and the position of the glider in meters.  In the mass column, type ``0'' and ``10'' in the first two cells.  Select these two cells, position the cursor at the lower left corner of the bottom cell until it turns into a lopsided square, and drag down the column to fill it with a series up to 60 grams.  The running yellow box shows how far the series continues.

\item Fill down the next column with the force values.  Remember how to do this?  Type ``\url{=A3*9.8/1000}'' in the cell next to the mass value of zero, being sure to use the correct cell designation corresponding to your spreadsheet (where we typed ``\url{A3}'' above).  Then fill down the forces next to the masses.

\item Now make the measurements.  Turn on the air blower, and help the glider come to equilibrium.  Add mases to the mass hanger, one at a time, and read the distance (in meters!) from the scale on the air track aligned with one corner of the glider.  On your spreadsheet, record the distance corresponding to the addition of each mass in the mass hanger.  Be sure that you use values corresponding to the entries in the mass column (including the mass of the hanger itself).  If you need to use different masses, change the mass entries; the forces will be recalculated instantly.

\item When you have filled in the distance column next to the force column, chart these variables against each other in Excel, and find the slope.  Here is a reminder of how this is done:
\squishlist
\item Select the cells with numbers in the force and distance columns.
\item In the ``Insert'' tab menu, select the ``Scatter'' option without connecting lines.  After clicking on this line-less scatter option, the chart should appear.
\item Within the ``Design'' tab of the ``Chart Tools'', in the ``Chart Layouts'' area, click on the left-most (and upper-most) layout option.  This will place titles on the chart that you can edit or delete.
\item Create your own title and labels for the axes.
\item In the ``Layout'' tab of the ``Chart Tools'', click on ``Trendline'' in the ``Analysis'' area.  Select ``More Trendline Options...''.  In the pop-up window, make sure ``Linear'' is selected, and check the box beside ``Display Equation on chart''.  Click on ``Close''.
\squishend
Convince yourself that the slope is \(1/k\) and not \(k\).  Record the value of \(k\) in the ``Data'' section.

\end{enumerate}

\subsection*{PROCEDURE PART 2: PLOTTING ENERGIES}

\begin{enumerate}[label=\arabic*.]

\item Unhook the mass hanger string. Hook up the smart pulley physically and virtually, in Capstone (the sensor is called ``Photogate with Pulley'').  Choose the ``Table \& Graph'' option in Capstone.  Choose ``Position (m)'' for the \(y\)-axis of the graph.  Add another graph by clicking the ``Add new plot area...'' button.  On this new graph, select ``Linear Speed (m/s)'' for the \(y\)-axis.  Add a third graph and plot ``Linear Acceleration ($m/s^2$) on the \(y\)-axis.

\item With the air blower on, pull the glider out, click ``Record'', let the glider oscillate several times, and click ``Stop''.  The velocity and acceleration graphs resemble the sinusoidal oscillations of a simple harmonic oscillator, but the position graph consists of a series of S-shaped curves increasing in \(y\) value.  This shape results because the smart pulley does not distinguish between the forward and reverse directions of motion; it merely counts the number of times the spokes block the photosensor and records the result as positive distance.  Thus, each S-shaped curve on the position graph is produced as the oscillator moves from one endpoint of its motion to the other.

\item Now record just half of an oscillation.  Pull the glider back, click ``Record'', then release the glider.  
\textbf{Make sure to click ``Record'' at least a second or two before releasing the glider.}
Click ``Stop'' as soon as you see the glider start to reverse directions, making sure not to stop it too early.  Your postion graph should be S-shaped and your velocity graph should look like an upside-down ``U''.

\item Click ``Select Measurement'' in the table and choose ``Position (m)''.  There should be about 25 values in the table.  Copy them over to a column in Excel.  Next, change the selected measurement in your table to ``Linear Speed (m/s)''.  Copy these values to a different column in Excel.

\item 
	Calculate the kinetic energy as a function of time using your velocity
	measurements, by setting up a formula in Excel.
%	You'll need to figure out how to calculate this by setting up a formula in excel.
	Recall that the kinetic energy of an object with mass $m$ and speed $v$
%	is $\text{KE}=\frac{1}{2}mv^2$.
	is $KE=(1/2)mv^2$.
%	To do this, set up the Excel formula, KE = \url{0.5*m*v^2}, 
%	where m is the mass of the glider and v is the velocity of the glider.  
	Now plot kinetic energy in Excel.  
	You should get a curve that looks like an upside-down ``U''.

\item Calculating the potential energy with the smart pulley is trickier.  First, as demonstrated above, the pulley does not distinguish between forward and backward motion, so we can look at only the first half-oscillation.  Second, when we pull the glider out and ``Start'' the distance measurement, the software assigns zero to the first distance measurement.  However, we want to assign zero to the equilibrium position.  In other words, we want to calculate \(PE = (1/2)k(x-x_0)^2\), with \(x_0\) the equilibrium position.  Use the average value of position as \(x_0\) and set up another Excel formula for calculating the potential energy (remember that k is the stiffness of the springs that you measured earlier).  Plot potential energy as a function of time.  It should be somewhat U-shaped.  

\item Create an Excel formula that calculates the total energy (kinetic + potential) as a function of time.  Plot all three of these (kinetic, potential and total) on the same graph.  To add a data set to a graph, right click on it and click ``Select Data...'', then click the ``Add'' button.  Clear the ``Series Y values:'' box and then select the column you want to add.  Then click ``OK'' two times.  The new data set should appear in the graph.

\item Once all three data sets are plotted on the same graph, check that it makes sense physically.  The total energy of the system is constant in theory.  However, you may see a slight reduction in total energy as time passes.  This is because energy is slowly lost due to friction and wind resistance.  Label the axes of your graph and show it to your TA.
Your graph should look something like the figure below:
\begin{center} \includegraphics*[width=0.6\textwidth]{imgs/6labs/6Alab/6Aexp4/excel-graph.png} \end{center}

\end{enumerate}

\subsection*{DATA}

\vphantom{.}
\squishlist

\item (Initial Setup, step 2) Mass of glider = \ul{~~~~~~~~~~~~~~~~~~~~~~~~~~~~~~~~~~~~~~~~~~~~~}

\item (Procedure Part 1, step 4) Spring constant \(k\) = \ul{~~~~~~~~~~~~~~~~~~~~~~~~~~~~~~~~~~~~~~~~~~~~~}

\item (Procedure Part 2, step 6) Equilibrium position \(x_0\) = \ul{~~~~~~~~~~~~~~~~~~~~~~~~~~~~~~~~~~~~~~~~~~~~~}

\squishend

\subsection*{ADDITIONAL CREDIT (3 mills)}

The glider-spring system is a simple harmonic oscillator.  You will study the physics of such systems later in the course. 
A crucial feature of such systems is that their motion is cyclic,
repeating itself with a certain frequency.

First devise a method and obtain the frequency of oscillation from your Capstone measurements (or make new measurements).
Next use your values of \(k\) and \(m\) to check your result against the following formula for the frequency of oscillation:
\begin{align*} f = \frac{1}{2\pi} \sqrt{k/m}. \end{align*}

Report the results with experimental error.
