\fancyhead[C]{Physics 6A Lab \rule[-1ex]{0pt}{0pt}\(\mid\) Experiment 5}

\vspace{-5ex}\part*{Momentum and Impulse}

\subsection*{APPARATUS}

\noindent\textit{Shown in the picture below:}
\squishlist
\item Air track
\item Glider with bumper and flag
\item Photogate
\item Force sensor
\squishend
\begin{center} \includegraphics*[width=0.6\textwidth]{imgs/6labs/6Alab/6Aexp5/glider_tack_main_white_text.jpg} \end{center}
\noindent\textit{Not shown in the picture above:}
\squishlist
\item Computer and Pasco interface
%\item Set of masses in 50-g increments to 600 g
\item Vernier calipers or meter stick to measure glider flag
\item Scale and weight set
\squishend

\subsection*{THEORY}

Newton's Second Law tells us that the net force acting on an object is equal to the object's mass multiplied by its acceleration: \(\textbf{F}_{\textrm{net}} = m\textbf{a}\).  Using \(\textbf{a} = \textrm{d}\textbf{v}/\textrm{d}t\), where \(\textbf{v}\) is the object's velocity, we can rewrite this law as
\begin{align} \textbf{F}_{\textrm{net}} =  m\,\textrm{d}\textbf{v}/\textrm{d}t. \end{align}
Newton himself believed that this relation should also account for the possibility that the mass is varying:
\begin{align} \textbf{F}_{\textrm{net}} = \textrm{d}(m\textbf{v})/\textrm{d}t. \end{align}
Examples of varying masses include rain falling into a rolling open box car and a rocket expelling gases.  The above equation can be rewritten as
\begin{align} \textbf{F}_{\textrm{net}} = \textrm{d}\textbf{p}/\textrm{d}t, \label{6Aexp5_eqn_1} \end{align}
where \(\textbf{p} = m\textbf{v}\) is the \textit{momentum} of the object.  Eq.~\ref{6Aexp5_eqn_1} is the most general definition of force: the change of momentum with time.  If we write it as a differential equation,
\begin{align} \textrm{d}\textbf{p} = \textbf{F}_{\textrm{net}} \textrm{d}t, \end{align}
and integrate with respect to time, then Eq.~\ref{6Aexp5_eqn_1} becomes
\begin{align} \Delta \textbf{p} = \textbf{p}_2 - \textbf{p}_1 = \int \textbf{F}_{\textrm{net}}\,\textrm{d}t. \label{6Aexp5_eqn_2} \end{align}
The right side of Eq.~\ref{6Aexp5_eqn_2} is known as the \textit{impulse}, and the left side is the change in momentum.  The notion of impulse is often associated with a force that acts for a short period of time.  Examples of such forces include a bat hitting a ball and the impact between two objects moving at relatively high speeds.

In this experiment, you will verify Eq.~\ref{6Aexp5_eqn_2} by allowing a glider on an air track to pass through a photogate and strike a force sensor.  The sensor allows you to measure the force on the glider as a function of time.  This time interval is relatively short, so the impulse approximation is valid.  The velocity of the glider is measured when it crosses the photogate, just before and just after the collision.  These two velocity measurements, along with knowledge of the glider's mass, allow you to calculate the change in momentum (i.e., the left side of Eq.~\ref{6Aexp5_eqn_2}).  The sensor generates a force-versus-time curve on the computer, which can be integrated to obtain the impulse (i.e., the right side of Eq.~\ref{6Aexp5_eqn_2}).  The glider has a foam bumper, so its collision with the force sensor is \textit{inelastic}.  In other words, kinetic energy is not conserved during the collision, but the change in momentum is still equal to the impulse.


\subsection*{PROCEDURE}

\begin{enumerate}[label=\arabic*.]

\item Level the air track carefully.  Mount the force sensor horizontally on the vertical rod of the ring stand at the end of the track so the glider bumper will strike the sensor as the glider moves down the track.  
Set up the photogate so the glider flag clears the gate by a few centimeters before the bumper strikes the sensor.  Refer to the picture below.
Note that while the sensor has two interchangeable end parts (a screw hook for
hanging objects from the force sensor, and a rubber bumper
for the impulse measurements in the actual experimental runs),
we will only be using the rubber bumper in this experiment.

\begin{center} \includegraphics*[width=0.6\textwidth]{imgs/6labs/6Alab/6Aexp5/gif_still.jpg} \end{center}
The glider then bounces off the sensor and passes through the photogate again.  In an experimental run, you should record the two photogate velocity readings and the force-versus-time curve from the sensor.

\item Weigh the glider, and record its mass (in kilograms) in the ``Data'' section.

\item Open Capstone and choose ``Table \& Graph''.  Under ``Hardware Setup'', click on Channel 1 and choose ``Photogate''.  Then click on Channel A and choose ``Force Sensor''.

\item We want to set up the photogate to measure the velocity of the glider.
\begin{enumerate}[label=\alph*.]
\item Click on ``Timer Setup'' and then click ``Next'' twice.
\item Select ``One Photogate (Single Flag)''.  Make sure ``Speed'' is selected and click next.
\item Make sure the flag width is entered in the white box (0.05 m).  Then click ``Next'' and then ``Finish''.
\end{enumerate}

\item Click on ``Select Measurement'' in one of your table columns and select ``Speed (m/s)''.  Click on the \(y\)-axis of the graph and choose ``Force (N)''.

\item At the bottom of the screen, set the sampling rate to 2000 Hz.  Note that the computer will then take a force reading every 1/2000, or 0.0005, second.

\item Turn on the air track, and push the ``Tare'' button on the side of the force sensor to zero its readings.  Click ``Record'' and send the glider down the track so it passes through the photogate, strikes and bounces off the force sensor, and crosses the gate again.  Click ``Stop''.  Your table should show two velocity readings at the top.  The second velocity value is smaller since the collision is inelastic.

\item Zoom in on the part of the graph where the glider hit the force sensor.  Your data should look something like that of the image below.


	\todo[inline]{Do we even want to mention the long list of voltage readings?
		If we do, we should probably explain more carefully in lieu of a
		calibration step.
	}


\begin{center}
\todo[inline]{Figure~\ref{fig:impulse-curve}, named \protect\url{MI_graph1_fix3.jpg},
needs to be updated to replace Voltage Ch A with Force Ch A.}
\includegraphics*[width=0.9\textwidth]{imgs/6labs/6Alab/6Aexp5/impulse.png}\label{fig:impulse-curve}\end{center}

\item Use the ``Display are under active data'' button to calculate the area under the curve.  Use the ``Highlight range of points...'' button to select only the data of interest.

\item Use the two speed recordings in your table to calculate the change in momentum from this collision.  When you calculate the change in momentum from \(mv_1\) to \(mv_2\), should you add or subtract these two numbers?  Pay attention to the direction the glider is moving and remember that the photogate measures speed, not velocity.

\item Have each lab partner make three measurements with different glider speeds to check the relation \(\Delta\textbf{p} = \int \textbf{F}_{\textrm{net}} \textrm{d}t\).  


\item In the ``Data'' section, record the three values of change in momentum and impulse, as well as the percentage difference between each set of values.

\end{enumerate}

\textbf{\textit{Procedure}}

\begin{enumerate}[start=2]

\item Mass of glider (kg) = \ul{~~~~~~~~~~~~~~~~~~~~~~~~~~~~~~~~~~~~~~~~~~~~~}

\setcounter{enumi}{11}
\item In the table below, enter your measured changes in momentum, impulses, and the percentage differences.
\begin{center} \includegraphics*[width=0.6\textwidth]{imgs/6labs/6Alab/6Aexp5/M_Itable_sm.png} \end{center}

\end{enumerate}

\subsection*{ADDITIONAL CREDIT (5 mills)}

Capstone performs a summation algorithm to calculate the area under a curve.  You can do this same technique using Excel.  In this experiment, you need to perform a sum over your force data to calculate the impulse.
	The impulse is the area under the force-versus-time curve.  It can be approximated using a Riemann sum:
\begin{align} \int \textbf{F}_{\textrm{net}} \textrm{d}t = \sum \textbf{F}_i \,\Delta t_i = \Delta t \sum F_i, \end{align}
since the time intervals \(\Delta t_i = 0.0005\) second are all equal.  You can obtain the force sum with an Excel function such as ``=Sum(b3..b147)''.  Perform this calculation on one of your data sets and compare to the results from Capstone.  You will need to make a table of force measurements and copy it over to Excel.