% filename: PhysMan.tex

%% FOR Sans Serif UNCOMMENT THE FOLLOWING TWO LINES
%\usepackage[T1]{fontenc}
%\renewcommand*\familydefault{\sfdefault} %% Only if the base font of the document is to be sans serif
%% (To get normal Serif font again, comment the two lines)


%----------------------------------------------------------------
% Attempts at changing the font to sans serif style
%----------------------------------------------------------------

% http://www.tug.dk/FontCatalogue/sansseriffonts.html

%%\usepackage[light,math]{iwona}
%%\usepackage[T1]{fontenc}

%%\usepackage{cmbright}
%%\usepackage[T1]{fontenc}

% Works:
%\usepackage[T1]{fontenc}
%\renewcommand*\familydefault{\sfdefault} %% Only if the base font of the document is to be sans serif

%%\usepackage[default]{sourcesanspro}
%%\usepackage[T1]{fontenc}



%----------------------------------------------------------------
% Other style options
%----------------------------------------------------------------

\usepackage{fullpage}
\usepackage{graphicx}
\usepackage{amsmath}
\usepackage{amssymb} % Math symbols such as \boldsymbol

\usepackage{fancyhdr}
\pagestyle{fancy}
%\fancyfoot[C]{\thepage}
\usepackage[headsep=1.5cm]{geometry} % Seperate header and main body by
		% headsep -- prevents first line from overlaying on top of header

\usepackage[noams,squaren,Gray]{SIunits} % Standardizes quantity and SI unit spacing
\usepackage{soul} % for an underline macro that wraps
 \setul{0.2ex}{.4pt}% underline 0.2ex below contents
\usepackage{enumitem} % for enumerate options
\usepackage{verbatim} % Allows display of almost any sequence of symbols, printed in typewriter font; and provides comment environment
\usepackage[obeyspaces,spaces]{url} % Same as verbatim but allows line breaks and spaces
\usepackage{parskip}
 \setlength{\parindent}{0em}
 \setlength{\parskip}{2.2ex}

%\usepackage[compact]{titlesec}
%\titlespacing{\section}{0pt}{*1ex}{*0.5ex}
%\titlespacing{\subsection}{0pt}{*0}{*0}
%\titlespacing{\subsubsection}{0pt}{*0}{*0}
\usepackage{titlesec}
\titlespacing{\subsection}{0pt}{*2.5}{*0.1}



%----------------------------------------------------------------
% Macros
%----------------------------------------------------------------

% Scientific Notation and Units
% =============================

% scientific notation without units
\providecommand{\sn}[2]{ {#1} \mkern-1.5mu\times\mkern-3mu 10^{#2} }
% scientific notation with units
\providecommand{\snu}[3]{ \unit{\sn{#1}{#2}}{#3} }

% create compact or ``squished'' lists with sublists:
\newcommand{\squishlist}{
   \vspace{-2.2ex}
   \begin{list}{$\bullet$}
    { \setlength{\itemsep}{0pt}      \setlength{\parsep}{3pt}
      \setlength{\topsep}{3pt}       \setlength{\partopsep}{0pt}
      \setlength{\leftmargin}{1.5em} \setlength{\labelwidth}{1em}
      \setlength{\labelsep}{0.5em} } }
\newcommand{\squishend}{
    \end{list}  }

% struts for spacing in tabulars/tables
\newcommand{\tstrut}{\rule{0pt}{2.5ex}}
\newcommand{\bstrut}{\rule[-1ex]{0pt}{0pt}}
\newcommand{\Tstrut}{\rule{0pt}{2.9ex}}
\newcommand{\Bstrut}{\rule[-1.4ex]{0pt}{0pt}}
