% Lab 6 Biceps Muscle Model

\documentclass{article}
\usepackage{textcomp}
\textwidth=420pt
\textheight=625pt
\voffset=-50pt
\hoffset=-25pt
\pagestyle{empty}

\begin{document}
\part*{Lab 6 Biceps Muscle Model}
TA: Chen-Hsuan Hsu\\
\vspace{-0.3in}
\section*{Theory}
\begin{itemize}
\item According to the Laws of Statics, all the forces acting on the forearm must satisfy 
\begin{equation}\label{eq1}
B = (H+ \frac{W}{2}) \frac{R}{(r \cos \alpha)},
\end{equation}
\begin{equation}\label{eq2}
A = (\frac{R}{r}-1)H+(\frac{R}{2r}-1)W,
\end{equation}
where $B$ is the force from the biceps muscle, $A$ is the humerus force from the upper arm, $W$ is the weight of forearm, $H$ is the weight of hand, $R$ is the moment arm of $H$, and $r$ is the moment arm of $B$. 

\end{itemize}

\section*{Experimental Process}

\subsection*{Procedure}
Skip step (1)-(4) in the procedure.

\begin{enumerate}

%\item Remove the forearm bar completely. The biceps tension-compression gauge is now suspended vertically so you can hang masses from it. Use the small keeper ring with the thumbscrew on the short right-angle section to prevent the masses from falling off. Notice that you can zero the scale by rotating the knurled ring at the top of the gauge.

%\item Hang one to four 1-kg masses from the gauge in increments of 1 kg. How accurately could the scale be read if there were no friction in the gauge? Record it, Estimated error $=\underline{~~~~~~~~~~~~~~~~~~~~~}kg$.

%\item With the heavier masses, there may be some friction in the gauge which brings it to rest over a small interval of readings. Move the mass hanger up and down near its equilibrium position, and record the maximum and minimum readings at which the pointer sticks. The average of these two values is a good estimate of the "true" mass, and their difference is a good estimate of the experimental error. Record the actual mass, maximum mass, minimum mass, average ("true") mass, and difference in mass ("error") in the chart below. State whether you observe any systematic error in the scale (i.e., whether the scale is off by a constant ratio).

%\begin{tabular}{|c|c|c|c|c|}
%\hline
%Bicepts  & one 1-kg mass  & two 1-kg masses & three 1-kg masses & four 1-kg masses \\
%\hline
%actual mass (kg) & 1 & 2 & 3 & 4 \\
%\hline
%maximum mass (kg) &  &  &  & \\
%\hline
%minimum mass (kg) &  &  &  & \\
%\hline
%average mass (kg) &  &  &  & \\
%\hline
%difference in mass (kg) &  &  &  & \\
%\hline
%\end{tabular}


%\item Repeat parts (2) and (3) with the upper-arm tension-compression gauge. Notice that the upper-arm gauge has a stiffer spring and a different scale than the biceps gauge. To hang masses from this gauge, you will need to invert the apparatus in its holding clamps. You may also need to rotate the gauge in the clamps so you can see its readings. Have your lab partner hold the biceps gauge out of the way, or tie it back with a piece of string. Adjust the knurled ring on the upper-arm gauge so the scale reads zero. Perform the same set of measurements as in parts (2) and (3), and record your results below. Estimated error $=\underline{~~~~~~~~~~~~~~~~~~~~~}kg$.

%\begin{tabular}{|c|c|c|c|c|}
%\hline
%Upper Arm & one 1-kg mass  & two 1-kg masses & three 1-kg masses & four 1-kg masses \\
%\hline
%actual mass (kg) & 1 & 2 & 3 & 4 \\
%\hline
%maximum mass (kg) &  &  &  & \\
%\hline
%minimum mass (kg) &  &  &  & \\
%\hline
%average mass (kg) &  &  &  & \\
%\hline
%difference in mass (kg) &  &  &  & \\
%\hline
%\end{tabular}


\item[(5)] Measure the distance between the elbow hole and each of the five biceps attachment holes ($r_{1}$, $r_{2}$, ... $r_{5}$), as well as the distance between the elbow hole and the mass hanger hole at the "hand" end of the forearm bar, $R$. Record these values:\\
\begin{tabular}{|c|c|c|c|c|c|}
\hline
$r_{1}$ & $r_{2}$ & $r_{3}$ & $r_{4}$ & $r_{5}$ & R\\
\hline
  &  &  &  & & \\
\hline
\end{tabular}


\item[(6)] Measure the mass of the forearm bar $M_{fore}$ without the upper-arm attachment piece, and record this value and calculate the force $W$: $M_{fore}=\underline{~~~~~~~~~~~~~~~~~~~~~}kg,~W=M_{fore}g=\underline{~~~~~~~~~~~~~~~~~~~~~}N$.

\item[(7)] Adjust the knurled rings of the gauge so they read zero with no weight attached. Arrange the apparatus as shown in Figure 1, with the biceps gauge attached to the farthest hole from the elbow, which is approximately \underline{12 cm} from the hole through which the upper-arm gauge passes. When clamping the forearm gauge, make sure the pointer can move freely through its range and is not obstructed by the clamp jaws. Use the small keeper ring with the thumbscrew on the short right-angle section to retain the biceps gauge in position. Attaching the 50-g mass hanger to the end of the forearm bar, add masses in increments of 100 g up to 550 g, and record the readings of the two gauges below. 

Since there may be some friction in the gauge which brings it to rest over a small interval of readings, you need to move the mass hanger up and down near its equilibrium position, and record the maximum and minimum readings at which the pointer sticks. The average of these two values is a good estimate of the "true" force, and a half of their difference is a good estimate of the experimental error. Record the maximum reading, minimum reading, average reading, and a half of the difference in readings ("error") in the chart below. 

As you change the masses in the "hand," adjust the position of the upper-arm attachment point at the "elbow" so the bar comes to rest in a horizontal position. You may need to push up forcefully on the bar when using the heavier masses, compressing the upper-arm gauge before tightening the attachment.

\begin{tabular}{cc}
\begin{tabular}{c}
\\
\\
Biceps gauge:\\
\\
\\
\\
Upper-arm gauge:\\
\\
\\
\\
\end{tabular}
 & 
\begin{tabular}{|c|c|c|c|c|c|}
\hline
 Total mass, $m$ & 150 g & 250 g & 350 g & 450 g & 550 g \\
\hline
 $H=mg$ (N) &  &  &  &  &  \\
\hline
 maximum reading &  &  &  &  &  \\
\hline
 minimum reading &  &  &  &  &  \\
\hline
 average reading, $B$ &  &  &  &  &  \\
\hline
 (difference in readings)/2, $\Delta B$ &  &  &  &  &  \\
\hline
 maximum reading &  &  &  &  &  \\
\hline
 minimum reading &  &  &  &  &  \\
\hline
 average reading, $A$ &  &  &  &  &  \\
\hline
(difference in readings)/2, $\Delta A$ &  &  &  &  &  \\
\hline
\end{tabular}
\end{tabular}

If you find that there seems to be an unusual amount of friction in your scale readings, check that the scales are not twisted in their clamps. All persons doing experiments in the real world soon realize that nature can be difficult, and not everything works as it is supposed to or according to the sample instructions. %This principle has been canonized in variations of Murphy's Law: "If anything can go wrong, it will," "Nature sides with the hidden flaw," etc. 
Throughout this lab series, you will often need to use common sense and resort to your own ingenuity to get through the parts that don't seem to work quite right. 

\item[(8)] With a total mass of 150 g hanging from the end of the forearm bar, take readings with the biceps gauge attached to the various holes, which are approximately 4, 6, 8, and 12 cm from the elbow. (The hole at 3 cm is difficult to use, so you DO NOT need to do it.) Each time you adjust to a new position, make sure the bar is horizontal. For certain biceps attachment positions (e.g., 4 cm from the elbow), you may find it necessary to move the elbow attachment point a considerable distance along the upper-arm guage rod, forcefully compressing the gauge to keep the bar horizontal. For each $r$, obtain the angle $\alpha$ from $\tan \alpha = x/y$ (you already know the value of x, but it will need to extend the lines of the gauge rods to determine the value of y, see p56). Record your results:

\begin{tabular}{|c|c|c|c|c|}
\hline
distance from elbow, $r=x$ & 4 cm & 6 cm & 8 cm & 12 cm\\
\hline
%Length of the extended upper-arm, 
$y$ &  &  &  &  \\
\hline
$\tan \alpha=x/y$ &  &  &  & \\
\hline
$R/(r\cos \alpha)$ &  &  &  & \\
\hline
maximum reading &  &  &  & \\
\hline
minimum reading &  &  &  & \\
\hline
average reading, $B$ &  &  &  & \\
\hline
 (difference in readings)/2, $\Delta B$ &  &  &  &  \\
\hline
\end{tabular}

\item[(9)] In Excel, plot a $B$-versus-$H$ graph using the table in (7). Label the axes with units. Add error bars by double clicking the data points in the graph (see p56) and choosing "Custom" to select the data of error bars. Fit the trend line to verify Eq. (\ref{eq1}).% construct the straight line predicted by Eq. (\ref{eq1}). 
Also, make another curve of $A$-versus-$H$ on the same graph. Add error bars, and fit the trend line to verify Eq. (\ref{eq2}).%construct the straight line predicted by Eq. (\ref{eq2}). 
Label the two curves, and title the graph.

\item[(10)] Plot a graph of the biceps force $B$ as a function of $R/(r \cos \alpha)$ for the various hole positions $r$ in Excel using the table in (8).  Label the axes, add error bars, and fit the trend line to verify Eq. (\ref{eq1}).% construct the straight line predicted by Eq. (\ref{eq1}).

\end{enumerate}

\subsection*{Question (2 mills)} 
Skip question 2

\begin{enumerate}
\item What is the largest value of the horizontal force, P, that would have arisen in your measurements? Hint: You may need the formula for the horizontal force.

%\item People who work out regularly can curl 25 to 50 lbs with one hand. The moment arm is approximately 5 cm. How large a force would the biceps muscle be exerting when one curls 50 lbs?

\end{enumerate}

\end{document} 
