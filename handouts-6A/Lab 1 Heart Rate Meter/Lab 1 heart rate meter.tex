% Lab 1 Heart Rate Meter

\documentclass{article}
\usepackage{textcomp}
%\textwidth=450pt
%\textheight=720pt
%\voffset=-70pt
%\hoffset=-50pt
\pagestyle{empty}

\begin{document}
\part*{\LARGE{Physics 6A: Lab 1 Heart Rate Meter}}
TA: Chen-Hsuan Hsu\\
\vspace{-0.3in}
\section*{Introduction}
\begin{itemize}
\item For each lab, the TA will distribute the handouts to help students to finish the lab more efficiently. You may follow the handouts during the lab, and check your lab book for more details of the procedure and theoretical background of the lab. You need to record your experimental results on BOTH your lab workbook and the handouts. When you finish the lab, write down the names of all members in your group on the handout. The TA will collect the handouts and check you off. 

\item The goal of lab 1 is to introduce you to computer acquisition of data and the Data Studio control program.


\end{itemize}

\section*{Experimental Process}

\subsection*{Procedure}
\begin{enumerate}
\item Plug the heart rate sensor into channel A on the interface box, and turn on the signal interface. On the computer screen, {\bf double click} on the \fbox{Data Studio} icon to bring up the program. In the new window, click \fbox{Create Experiment}. Then, you will see a "Experiment Setup" window at the upper right as the picture on page 4 in the lab book. On the left side of the screen is a column with "Data" at the top, followed by a list of possible ways to display data.

\item {\bf Click} on the channel A in the "Experiment Setup" window to select the sensor. {\bf Scroll} the list of sensors until you find the heart rate sensor with the red heart icon, and {\bf double click} on it . A symbol of the sensor will appear in the picture, connected to the channel A (page 5).

\item From the "Displays" column at the lower left of the screen, {\bf drag} the \fbox{3.14 Digits} symbol over to the heart symbol in the "Experiment Setup" window. A digits window appears (page 5).

\item {\bf Drag} the digits window to a clear area of the screen. The window should look like the picture on page 5. From the "Data" column, {\bf Drag} "Heart Rate, CH A (beats/min)" over the digits window.

\item Clip the heart sensor to your ear lobe. To see your heart rate in the digits window, {\bf click} the \fbox{Start} button on the top tool bar. Observe the data for about one minute, and then {\bf click} \fbox{Stop}. You should see your resting heart rate (55-70 beats/min). The figure changes slightly every few second. If you do not see a clear, reasonable rate, try repositioning the ear clip or attaching it to your other ear.  

\item To see the sensor voltage produced by your heartbeat, {\bf drag} a graph symbol to the heart symbol. A graph with "No Data Selected" appears. {\bf Drag} the graph to a clear area of the screen. From the "Data" column, {\bf Drag} "Voltage, CH A (V)" over to the y-axis of the graph. The x-axis should change automatically to "Time". If you have already taken a data run to measure your heart rate with the digits window, you will see the voltage output data shown in the graph.

\item To observe your heart pulses in real time, {\bf click} \fbox{Start} to take a new data run. You will see new data taken in real time plotted in a different color on the graph. {\bf Click} \fbox{Stop} when you are finished. Again, if you do not see a clear graph, try repositioning the ear clip or attaching it to your other ear, and try this step again.

\item If you want to receive the additional credits, you can go ahead to do the additional part. Otherwise, you may ask the TA to check you off now. 

\end{enumerate}

\subsection*{Additional Part: Raising and lowering your heart rate (2mills)}

\begin{enumerate}

\item In this part, you will raise your heart rate to a moderately high value by exercising, and then produce a graph of heart rate as a function of time as the heart rate relaxes back to its resting mode. 

\item Drag a new graph over to the heart rate sensor symbol. (Please do not delete your previous data if you have not shown them to your TA.) This time, drag "Heart rate, CH A (beats/min)" over to the y-axis of the graph. 

\item Take off the sensor and exercise to raise your heart rate. You can go out and run around the building, or do jumping jacks in position. Raise your heart rate to 140 beats per minute or higher. \underline{Do not perform the exercise } \underline{if you have a health problem. Have your lab partner do it.}  After exercising, reattach the sensor, and click \fbox{Start} to record. Remain as still as possible while the recording is made for 5 minutes. Click \fbox{Stop} when you are finished.

\item If you do not obtain a smooth graph, get some more exercise and try again. (You may edit the graph using the graph settings window if you like. See page 7 for the details.) You can print out the graphs if you like to keep hard copies of your graphs. You can use the three-hole punch to punch this sheet, and clip it in your lab workbook.

\item When you have all the clear graphs and data shown on the screen, ask the TA to check you off. 

\end{enumerate}

\end{document} 
